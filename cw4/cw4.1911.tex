\documentclass[12pt, letterpaper, titlepage]{article}
\usepackage[left=3.5cm, right=2.5cm, top=2.5cm, bottom=2.5cm]{geometry}
\usepackage[MeX]{polski}
\usepackage[utf8]{inputenc}
\usepackage{graphicx}
\usepackage{enumerate}
\usepackage{amsmath}
\usepackage{amssymb}
\title{przepis na ciasto z rabarbarem i budyniem}
\author{Sandra Koślicka}
\date{Październik 2022}
\begin{document}
\maketitle

Ułamek w tekście $ \frac{1}{x} $\\
Oto równanie $ c^{2}=a^{2}+b^{2} $

\begin{equation}
\frac{1}{x^2}
\end{equation}
Oto równanie
\begin{equation}
c^{2}=a^{2}+b^{2}
\end{equation}
Indeks górny $$ x^{y} \ e^{x} \ 2^{e} \ A^{2 \times2} $$\\
Indeks dolny $$ x_y \ a_{ij} \ x_i $$\\
Oba indeksy $$ x_i^{2} \ x_{i^2}^{k_j}$$\\

$$\sqrt{ \frac{2^{n}}{2_n}} \neq \sqrt[\frac{1}{3}]{1+n} $$\\
$$\frac{2^{k}}{2^{k+2}} $$\\

$$\frac{2^{k}}{2^{k+2}} $$\\

$$\frac{x^2}{2^{(x+2)(x-2)^{3}}} $$\\

$$ log_2 2^8 \ =8 $$\\
$$\sqrt[3]{e^x-log_2 x}$$\\


\end{document}

